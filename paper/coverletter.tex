\documentclass[fromalign=right,backaddress=false]{scrlttr2} 
\usepackage[english]{babel}
\setkomavar{fromname}{\small Quentin Caudron} 
\setkomavar{fromaddress}{\small Department of Ecology and Evolutionary Biology \\ Princeton University \\ Princeton, NJ, 08648, USA} 
\KOMAoptions{foldmarks=off}
\let\raggedsignature\raggedright 

\begin{document} 
\begin{letter}{\small Editors \\ \small Journal of the Royal Society Interface} 
  \opening{Dear Editors,} 
  We are pleased to submit our manuscript, ``Predictability in a highly stochastic system~: measles in small populations'' for consideration at the Journal of the Royal Society Interface. The paper represents original research and is not under submission at any other journal.
  \vspace{0.3cm}
  
  The dynamics of highly contagious, immunising childhood infections such as measles are well characterised in many places around the world. Prior to the introduction of vaccination, measles was typically endemic, demonstrating regular oscillations in incidence. In small, isolated populations, however, the disease was frequently absent from the population, and, being introduced from abroad at a random time, would spread rapidly through the population, only to become soon extinct as the number of susceptible individuals became too low to sustain transmission. In this study, we considered the problem of predicting the temporal evolution of incidence dynamics for these isolated, small populations with sudden, rapid epidemics. 
    \vspace{0.3cm}
  
  Using a discrete-time SIR model, we inferred the dynamics of the number of susceptible individuals, and the reporting rate for the incidence of measles in six populations from prevaccination Iceland, Bornholm, and the Faroe Islands. Then, we simulated the evolution of measles incidence over time, generating predicted time-series from only the first data point in an epidemic. Finally, we compared both the predicted dynamics and the estimated final epidemic sizes from our simulations to the data. 
      \vspace{0.3cm}
      
  Our results show that, despite the sparsity of the data, the lack of any significant seasonal forcing, and reduced levels of population mixing, a strong signal of SIR-like dynamics exists in the time-series. Using this methodology, we are able to predict the final sizes of epidemics with a significant level of certainty. 
    \vspace{0.3cm}

We believe that this is the first study to apply time-series methods to these famous small island datasets. These results may have implications for the control of future epidemics in locations where the disease is subendemic. Given the highly interdisciplinary nature of the methods used in this manuscript, we believe it is likely to be of broad interest to the readership of the Journal of the Royal Society Interface.
    \vspace{0.3cm}

We thank you in advance for considering our paper, and look forward to a favourable response.


  
  
  
  \closing{Sincerely,}
\end{letter} 

\end{document}