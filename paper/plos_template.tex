% Template for PLoS
% Version 1.0 January 2009
%
% To compile to pdf, run:
% latex plos.template
% bibtex plos.template
% latex plos.template
% latex plos.template
% dvipdf plos.template

\documentclass[10pt]{article}

% amsmath package, useful for mathematical formulas
\usepackage{amsmath}
% amssymb package, useful for mathematical symbols
\usepackage{amssymb}

% graphicx package, useful for including eps and pdf graphics
% include graphics with the command \includegraphics
\usepackage{graphicx}

% cite package, to clean up citations in the main text. Do not remove.
\usepackage{cite}

\usepackage{color} 

% Use doublespacing - comment out for single spacing
%\usepackage{setspace} 
%\doublespacing


% Text layout
\topmargin 0.0cm
\oddsidemargin 0.5cm
\evensidemargin 0.5cm
\textwidth 16cm 
\textheight 21cm

% Bold the 'Figure #' in the caption and separate it with a period
% Captions will be left justified
\usepackage[labelfont=bf,labelsep=period,justification=raggedright]{caption}

% Use the PLoS provided bibtex style
\bibliographystyle{plos2009}

% Remove brackets from numbering in List of References
\makeatletter
\renewcommand{\@biblabel}[1]{\quad#1.}
\makeatother


% Leave date blank
\date{}

\pagestyle{myheadings}
%% ** EDIT HERE **


%% ** EDIT HERE **
%% PLEASE INCLUDE ALL MACROS BELOW

%% END MACROS SECTION

\begin{document}

% Title must be 150 characters or less
\begin{flushleft}
{\Large
\textbf{Measles in small populations : predictability in highly stochastic systems}
}
% Insert Author names, affiliations and corresponding author email.
\\
Q. Caudron$^{1,\ast}$, 
A. S. Mahmud$^{2}$, 
C. J. E. Metcalf$^{1}$,
B. T. Grenfell$^{1}$
\\
\bf{1} Department of Ecology and Evolutionary Biology, Princeton University, Princeton, NJ, 08544, USA
\\
\bf{2} Office of Population Research, Woodrow Wilson School of Public and International Affairs, Princeton University, Princeton, NJ, 08544, USA
\\
$\ast$ E-mail: qcaudron@princeton.edu
\end{flushleft}













% Please keep the abstract between 250 and 300 words
\section*{Abstract}

Blah


% Please keep the Author Summary between 150 and 200 words
% Use first person. PLoS ONE authors please skip this step. 
% Author Summary not valid for PLoS ONE submissions.   
%\section*{Author Summary}














\section*{Introduction}

Measles is a highly contagious and strongly immunizing infection of the respiratory system. Due to its extreme transmissibility, its epidemiology is conditional on the birth of susceptible individuals. As such, the temporal dynamics of measles are typically strongly oscillatory, driven seasonally by the increased contact rate amongst children during school periods, assuming the population is large enough to sustain the disease (Black FL, 1966, JTB 11). These dynamics have been well studied (Grenfell papers, others), and many modelling efforts have successfully explained the biennial cycle exhibited in prevaccination records of measles incidence in Europe and elsewhere (papers ?). 

In small populations, where the number of individuals is much smaller than the critical community size required to support an endemic infection, however, the dynamics of measles cases are vastly different. Susceptible individuals accumulate when measles is absent; then, driven by stochastic importation, an epidemic may sweep through a large fraction of the susceptible population very quickly, only to go extinct abruptly as susceptibles become depleted. This results in very sharp, spiky epidemics, the timing of which may be impossible to predict, but the size and duration of which may be a function of historical data. 

In this paper, we address the question of predictability of measles epidemics in small populations, based on records of past incidence and on demographic data. We present data on the demographics and disease incidence in prevaccination-era Bornholm, the Faroe Islands, and four districts in Iceland. 

reconstruct the dynamics of susceptible individuals and infer the rate of reporting of cases using the TSIR model (Finkelstadt and Grenfell 2000) in prevaccination 
 


sizes are, by training a model on records of past epidemics and demographic data. Using the TSIR model (Finkelstadt and Grenfell, 2000), we explore the dynamics of measles in prevacinnation Bornh










% Results and Discussion can be combined.
\section*{Results}

\subsection*{Subsection 1}

\subsection*{Subsection 2}












\section*{Discussion}










% You may title this section "Methods" or "Models". 
% "Models" is not a valid title for PLoS ONE authors. However, PLoS ONE
% authors may use "Analysis" 
\section*{Materials and Methods}











% Do NOT remove this, even if you are not including acknowledgments
\section*{Acknowledgments}









%\section*{References}
% The bibtex filename
\bibliography{template}

\section*{Figure Legends}
%\begin{figure}[!ht]
%\begin{center}
%%\includegraphics[width=4in]{figure_name.2.eps}
%\end{center}
%\caption{
%{\bf Bold the first sentence.}  Rest of figure 2  caption.  Caption 
%should be left justified, as specified by the options to the caption 
%package.
%}
%\label{Figure_label}
%\end{figure}


\section*{Tables}
%\begin{table}[!ht]
%\caption{
%\bf{Table title}}
%\begin{tabular}{|c|c|c|}
%table information
%\end{tabular}
%\begin{flushleft}Table caption
%\end{flushleft}
%\label{tab:label}
% \end{table}

\end{document}

